\section{Introduction}
Image inpainting is the technique of modifying an image in an undetectable
form. Some of the common cases image inpainting are:
\begin{enumerate*}[label=(\roman*)]
    \item the restoration of old photographs and damaged films
    \item the removal of unwanted text from an image
    \item the removal of an object in purpose
\end{enumerate*}. The existence of the image inpainting technique dates
back to the beginning of human art and we can see that the application of
image inpainting by computer saves a lot of time compared with hand-carfted
image inpaiting in museums.

There are alreay several researches on this topic. One kind of technique is
based on smoothing, \ie we inpaint that image based on the color of nearby
pixels. This method has an intrinsic problem of blurring since the color
chosed to inpaint the missing area are all based on the smoothing of nearby
colors. Another approach is kind of patch-based image inpainting, where
we choose a small patch around the edge of target area and propagate it into
target area. In this method, in contrast to the smoothing-based algorithm,
can better capture the texture of the image and will not lead to the problem
of blurring.

With regard to the patch-based method, every patch around the target region
are assigned a priority. When we propagate the patch into target region,
the patch with highest priority is chosed. In this paper, we propose an
efficient priority assginment strategy that can better capture the structure
of the whole image.
\textcolor{blue}{Kimi: TODO: add illustration of our proposal}
