\section{Related work and our contribution}
First we should make it clear that image inpainting and image denoising are
actually different tasks. In image inpainting, we aim to remove part of 
an image and a person needs to explicitly specify the area to be inpainted.
In constrast, as regard to image denoising, the task is to remove the 
``random'' noise of an image, which often does not need human supervison.
\subsection{Related work}
There are mainly two kinds of works on image inpainting: smoothing-based and
patch-based.

The research on image inpainting emerges on 2000, Marcelo Bertalmio \etal
\cite{siggraph00} first deals with the problem of image inpainting. Their
method is mainly based on intuition: propagate color information in the
direction of normal. However the limitation of the algorithm is that the
texture of its environment is unlikely to be reproduced.

Roth and Black \cite{cvpr05} later come up with a more general framework 
called \emph{Field of Experts} which is a Markov random field model. This
work relies on the result of Hinton \cite{neco02, nips02}, where Hinton discovers
that a factor in MRF can be modeled by a field of ``expert'' distributions.
In this work, they first learn the model and then apply the learned model
to Bertalmio's propagation method. Although this work in some sense captures
image structure, the blurring problem is also serious after our experiment. Later Bugeau\etal \cite{tip10} refined their result and optimize the execution time of the original method, but the problem of blurring is still there.

Another different way to tackle this problem is proposed by Criminisi \etal
 \cite{cvpr03,tip04}. They note that exemplar-based texture synthesis
 contains the essential process requred to replicate the structure of image.
 They introduce a priority for each exemplar and propagate according to
 the priority of each exemplar. This technique does not have the problem
 of blurring and can restore texture well, but may still fail for large
 object removal. Zongben Xu and Jian Sun also proposed a fix to the Criminisi's method in \cite{tip10sj}.

\subsection{Our contribution}
\textcolor{blue}{add our contribution}
