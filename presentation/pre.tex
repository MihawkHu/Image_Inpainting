\documentclass{beamer}

\mode<presentation> {

% The Beamer class comes with a number of default slide themes
% which change the colors and layouts of slides. Below this is a list
% of all the themes, uncomment each in turn to see what they look like.


\usetheme{default}
%\usetheme{AnnArbor}
%\usetheme{Antibes}
%\usetheme{Bergen}
%\usetheme{Berkeley}
%\usetheme{Berlin}
%\usetheme{Boadilla}
%\usetheme{CambridgeUS}
%\usetheme{Copenhagen}
%\usetheme{Darmstadt}
%\usetheme{Dresden}
%\usetheme{Frankfurt}
%\usetheme{Goettingen}
%\usetheme{Hannover}
%\usetheme{Ilmenau}
%\usetheme{JuanLesPins}
%\usetheme{Luebeck}
%\usetheme{Madrid}
%\usetheme{Malmoe}
%\usetheme{Marburg}
%\usetheme{Montpellier}
%\usetheme{PaloAlto}
%\usetheme{Pittsburgh}
%\usetheme{Rochester}
%\usetheme{Singapore}
%\usetheme{Szeged}
%\usetheme{Warsaw}

% As well as themes, the Beamer class has a number of color themes
% for any slide theme. Uncomment each of these in turn to see how it
% changes the colors of your current slide theme.

%\usecolortheme{albatross}
%\usecolortheme{beaver}
%\usecolortheme{beetle}
%\usecolortheme{crane}
%\usecolortheme{dolphin}
%\usecolortheme{dove}
%\usecolortheme{fly}
%\usecolortheme{lily}
%\usecolortheme{orchid}
%\usecolortheme{rose}
%\usecolortheme{seagull}
%\usecolortheme{seahorse}
%\usecolortheme{whale}
%\usecolortheme{wolverine}

%\setbeamertemplate{footline} % To remove the footer line in all slides uncomment this line
\setbeamertemplate{footline}[page number] % To replace the footer line in all slides with a simple slide count uncomment this line

\setbeamertemplate{navigation symbols}{} % To remove the navigation symbols from the bottom of all slides uncomment this line
}

\usepackage{graphicx}
\usepackage{booktabs}
\usepackage{amssymb}
\usepackage{color}



\AtBeginSection[]{
  \begin{frame}
  \vfill
  \centering
  \begin{beamercolorbox}[sep=8pt,center,shadow=true,rounded=true]{title}
    \usebeamerfont{title}\insertsectionhead\par%
  \end{beamercolorbox}
  \vfill
  \end{frame}
}


\title{Image Inpainting Software}

\author{Team 27: Yesheng Ma\\\hspace{0.9cm}Hu Hu\\\hspace{1.4cm}Yikai Zou}
\institute[SJTU]{}
\date{\today}

\begin{document}

\begin{frame}
\titlepage
\end{frame}



\section{Introduction}


\section{Existing Approaches}


\section{Proposed Ideas}
\subsection{Markov Random Field}

\subsection{Traditional Algorithm}


\section{Implementation}

\section{Demo}

\section{Conclusion and Future Work}

\section*{Reference}


%\begin{frame}
%\frametitle{Overview}
%\tableofcontents
%\end{frame}

%\section{Introduction}
%\begin{frame}
%\frametitle{Prisoner's Dilemma}
%Prisoner's Dilemma:\\
%\begin{itemize}[<+->]
%\item
%\begin{tabular}{|c|c|c|}
%\hline
%\hline
%    &{\color{red}Alice} Deny&{\color{red}Alice} Confess\\
%\hline
%{\color{blue}Bob} Deny& ({\color{blue}-1},{\color{red}-1}) & ({\color{blue}-9},{\color{red}0})\\
%\hline
%{\color{blue}Bob} Confess& ({\color{blue}0},{\color{red}-9}) & ({\color{blue}-6},{\color{red}-6})\\
%\hline
%\hline
%\end{tabular}
%\item
%{\color{blue}$S_B = \{ C, D\}$}  \qquad {\color{red}$S_A = \{ C, D\}$}
%\item
%{\color{blue}$u_B(D,{\color{red}D}) = -1$} \qquad {\color{red}$u_A({\color{blue}D},D) = -1$}
%\item
%{\color{blue}$u_B(D,{\color{red}C}) = -9$} \qquad {\color{red}$u_A({\color{blue}D},C) = 0$}\\
%{\color{blue}$u_B(C,{\color{red}D}) = 0 $} \text{ } \qquad {\color{red}$u_A({\color{blue}C},D) = -9$}\\
%{\color{blue}$u_B(C,{\color{red}C}) = -6$} \qquad {\color{red}$u_A({\color{blue}C},C) = -6$}
%\end{itemize}
%\end{frame}
%
%\begin{frame}
%\frametitle{What is a Non-zero Sum Game?}
%\begin{itemize}
%\item
%The sum of each player's gain or loss $\neq$ what they begin with.
%\item
%$\exists (s_1,s_2,...,s_n)\in S_1\times S_2 \times ... \times S_n$, $\sum_{i=1}^{n} u_i(s_1,s_2,...,s_n) \neq 0$
%\end{itemize}
%\end{frame}
%
%\begin{frame}
%\frametitle{Strict Domination}
%Assuming you are {\color{blue}Bob}, what should you do?\\
%\begin{itemize}[<+->]
%\item
%\begin{tabular}{|c|c|c|}
%\hline
%\hline
%    &{\color{red}Alice} Deny&{\color{red}Alice} Confess\\
%\hline
%{\color{blue}Bob} Deny& ({\color{blue}-1},{\color{red}-1}) & ({\color{blue}-9},{\color{red}0})\\
%\hline
%{\color{blue}Bob} Confess& ({\color{blue}0},{\color{red}-9}) & ({\color{blue}-6},{\color{red}-6})\\
%\hline
%\hline
%\end{tabular}
%
%\item
%What if {\color{red}Alice} chooses to \textbf{Deny}?
%\item
%Result: {\color{blue}Bob} is free and {\color{red}Alice} will spend 9 years.
%\item
%What if {\color{red}Alice} chooses to \textbf{Confess}?
%\item
%Result: {\color{blue}Bob} and {\color{red}Alice} will spend 6 years together.
%\item
%\textbf{In both cases, }{\color{blue}Bob}\textbf{ will definitely choose to confess.}
%\end{itemize}
%\end{frame}
%
%\begin{frame}
%\frametitle{Strict Domination}
%The process is like:\\
%\centering
%\begin{tabular}{|c|c|c|}
%\hline
%\hline
%    &{\color{red}Alice} Deny&{\color{red}Alice} Confess\\
%\hline
%{\color{blue}Bob} Deny& ({\color{blue}-1},{\color{red}-1}) & ({\color{blue}-9},{\color{red}0})\\
%\hline
%{\color{blue}Bob} Confess& ({\color{blue}0},{\color{red}-9}) & ({\color{blue}-6},{\color{red}-6})\\
%\hline
%\hline
%\end{tabular}
%\end{frame}
%
%\begin{frame}
%\frametitle{Strict Domination}
%The process is like:\\
%\centering
%\begin{tabular}{|c|c|c|}
%\hline
%\hline
%    &{\color{red}Alice} Deny&{\color{red}Alice} Confess\\
%\hline
%{\color{blue}Bob} Deny& ({\color{blue}-1},{\color{red}-1}) & ({\color{blue}-9},{\color{red}0})\\
%\hline
%{\color{blue}Bob} Confess& ({\color{blue}0},{\color{red}-9}) & ({\color{blue}-6},{\color{red}-6})\\
%\hline
%\hline
%\end{tabular}\\
%$\Downarrow$\\
%\begin{tabular}{|c|c|c|}
%\hline
%\hline
%    &{\color{red}Alice} Deny&{\color{red}Alice} Confess\\
%\hline
%{\color{blue}Bob} Confess& ({\color{blue}0},{\color{red}-9}) & ({\color{blue}-6},{\color{red}-6})\\
%\hline
%\hline
%\end{tabular}\\
%\end{frame}
%
%\begin{frame}
%\frametitle{Strict Domination}
%The process is like:\\
%\centering
%\begin{tabular}{|c|c|c|}
%\hline
%\hline
%    &{\color{red}Alice} Deny&{\color{red}Alice} Confess\\
%\hline
%{\color{blue}Bob} Deny& ({\color{blue}-1},{\color{red}-1}) & ({\color{blue}-9},{\color{red}0})\\
%\hline
%{\color{blue}Bob} Confess& ({\color{blue}0},{\color{red}-9}) & ({\color{blue}-6},{\color{red}-6})\\
%\hline
%\hline
%\end{tabular}\\
%$\Downarrow$\\
%\begin{tabular}{|c|c|c|}
%\hline
%\hline
%    &{\color{red}Alice} Deny&{\color{red}Alice} Confess\\
%\hline
%{\color{blue}Bob} Confess& ({\color{blue}0},{\color{red}-9}) & ({\color{blue}-6},{\color{red}-6})\\
%\hline
%\hline
%\end{tabular}\\
%$\Downarrow$\\
%\begin{tabular}{|c|c|c|}
%\hline
%\hline
%    &{\color{red}Alice} Confess\\
%\hline
%{\color{blue}Bob} Confess& ({\color{blue}-6},{\color{red}-6})\\
%\hline
%\hline
%\end{tabular}\\
%\end{frame}
%
%\begin{frame}
%\frametitle{Strict Domination}
%\begin{itemize}[<+->]
%\item
%If one of the player’s strategies is never the right thing to do, no matter what the opponents do, it is \textbf{Strictly Dominated}.
%\item
%Get rid of the strictly dominated strategies because \textbf{they won't happen}.
%\item
%This is called \textbf{iterated elimination of dominated strategies}.
%\end{itemize}
%\end{frame}
%
%
%\section{Mixed Strategy Example}
%\begin{frame}
%\frametitle{Bob and Alice}
%\begin{itemize}
%\item {\color{blue}Bob} and {\color{red} Alice} are students in some school.
%\item {\color{blue}Bob} loves {\color{red} Alice} but {\color{red} Alice} dont like {\color{blue}Bob}.
%\item The situation arises when they decide where to eat lunch.
%\end{itemize}
%\end{frame}
%
%\begin{frame}
%\frametitle{Restaurant}
%\begin{itemize}
%\item Restaurant No.1's food is awful.
%\item Restaurant No.3's food is better.
%\end{itemize}
%
%\end{frame}
%
%\begin{frame}
%\frametitle{Their Pay-off when eating}
%\begin{tabular}{|c|c|c|}
%\hline
%\hline
%    & {\color{red}Alice} go to No.3 & {\color{red}Alice} go to No.1\\
%\hline
%{\color{blue}Bob} go to No.3 & ({\color{blue}10},{\color{red}4}) & \\
%\hline
%{\color{blue}Bob} go to No.1 &  & \\
%\hline
%\hline
%\end{tabular}
%\end{frame}
%
%\begin{frame}
%\frametitle{Their Pay-off when eating}
%\begin{tabular}{|c|c|c|}
%\hline
%\hline
%    & {\color{red}Alice} go to No.3 & {\color{red}Alice} go to No.1\\
%\hline
%{\color{blue}Bob} go to No.3 & ({\color{blue}10},{\color{red}4}) & ({\color{blue}3},{\color{red}5})\\
%\hline
%{\color{blue}Bob} go to No.1 &  & \\
%\hline
%\hline
%\end{tabular}
%\end{frame}
%
%\begin{frame}
%\frametitle{Their Pay-off when eating}
%\begin{tabular}{|c|c|c|}
%\hline
%\hline
%    & {\color{red}Alice} go to No.3 & {\color{red}Alice} go to No.1\\
%\hline
%{\color{blue}Bob} go to No.3 & ({\color{blue}10},{\color{red}4}) & ({\color{blue}3},{\color{red}5})\\
%\hline
%{\color{blue}Bob} go to No.1 & ({\color{blue}0},{\color{red}10}) & \\
%\hline
%\hline
%\end{tabular}
%\end{frame}
%
%\begin{frame}
%\frametitle{Their Pay-off when eating}
%\begin{tabular}{|c|c|c|}
%\hline
%\hline
%    & {\color{red}Alice} go to No.3 & {\color{red}Alice} go to No.1\\
%\hline
%{\color{blue}Bob} go to No.3 & ({\color{blue}10},{\color{red}4}) & ({\color{blue}3},{\color{red}5})\\
%\hline
%{\color{blue}Bob} go to No.1 & ({\color{blue}0},{\color{red}10}) & ({\color{blue}7},{\color{red}0})\\
%\hline
%\hline
%\end{tabular}
%\end{frame}
%
%\begin{frame}
%\frametitle{How will they act?}
%\begin{tabular}{|c|c|c|}
%\hline
%\hline
%    & {\color{red}Alice} go to No.3 & {\color{red}Alice} go to No.1\\
%\hline
%{\color{blue}Bob} go to No.3 & ({\color{blue}10},{\color{red}4}) {\color{green}*}& ({\color{blue}3},{\color{red}5})\\
%\hline
%{\color{blue}Bob} go to No.1 & ({\color{blue}0},{\color{red}10}) & ({\color{blue}7},{\color{red}0})\\
%\hline
%\hline
%\end{tabular}
%\begin{itemize}
%\item
%If {\color{blue}Bob} claims in WeChat that he will go to No.3 and {\color{red}Alice} claims that she will go to No.3.
%
%\item What will happen?
%\end{itemize}
%\end{frame}
%
%\begin{frame}
%\frametitle{How will they act?}
%\begin{tabular}{|c|c|c|}
%\hline
%\hline
%    & {\color{red}Alice} go to No.3 & {\color{red}Alice} go to No.1\\
%\hline
%{\color{blue}Bob} go to No.3 & ({\color{blue}10},{\color{red}4})& ({\color{blue}3},{\color{red}5})  {\color{green}*}\\
%\hline
%{\color{blue}Bob} go to No.1 & ({\color{blue}0},{\color{red}10}) & ({\color{blue}7},{\color{red}0})\\
%\hline
%\hline
%\end{tabular}
%\begin{itemize}
%\item {\color{red}Alice} will choose to go to No.3 restaurant.
%\end{itemize}
%\end{frame}
%
%\begin{frame}
%\frametitle{How will they act?}
%\begin{tabular}{|c|c|c|}
%\hline
%\hline
%    & {\color{red}Alice} go to No.3 & {\color{red}Alice} go to No.1\\
%\hline
%{\color{blue}Bob} go to No.3 & ({\color{blue}10},{\color{red}4})& ({\color{blue}3},{\color{red}5})\\
%\hline
%{\color{blue}Bob} go to No.1 & ({\color{blue}0},{\color{red}10}) & ({\color{blue}7},{\color{red}0}) {\color{green}*}\\
%\hline
%\hline
%\end{tabular}
%\begin{itemize}
%\item Then {\color{blue}Bob} will choose to go to No.3 restaurant.
%\end{itemize}
%\end{frame}
%
%\begin{frame}
%\frametitle{How will they act?}
%\begin{tabular}{|c|c|c|}
%\hline
%\hline
%    & {\color{red}Alice} go to No.3 & {\color{red}Alice} go to No.1\\
%\hline
%{\color{blue}Bob} go to No.3 & ({\color{blue}10},{\color{red}4})& ({\color{blue}3},{\color{red}5}) \\
%\hline
%{\color{blue}Bob} go to No.1 & ({\color{blue}0},{\color{red}10}) {\color{green}*} & ({\color{blue}7},{\color{red}0})\\
%\hline
%\hline
%\end{tabular}
%\begin{itemize}
%\item Then {\color{red}Alice} will choose to go to No.3 restaurant.
%\end{itemize}
%\end{frame}
%
%\begin{frame}
%\frametitle{How will they act?}
%\begin{tabular}{|c|c|c|}
%\hline
%\hline
%    & {\color{red}Alice} go to No.3 & {\color{red}Alice} go to No.1\\
%\hline
%{\color{blue}Bob} go to No.3 & ({\color{blue}10},{\color{red}4}) {\color{green}*}& ({\color{blue}3},{\color{red}5})\\
%\hline
%{\color{blue}Bob} go to No.1 & ({\color{blue}0},{\color{red}10}) & ({\color{blue}7},{\color{red}0})\\
%\hline
%\hline
%\end{tabular}
%\begin{itemize}
%\item Then {\color{blue}Bob} will choose to go to No.3 restaurant.
%\item {\Large It is a circulation!!}
%\end{itemize}
%\end{frame}
%
%\begin{frame}
%\frametitle{What should they do?}
%\begin{tabular}{|c|c|c|}
%\hline
%\hline
%    & {\color{red}Alice} go to No.3 & {\color{red}Alice} go to No.1\\
%\hline
%{\color{blue}Bob} go to No.3 & ({\color{blue}10},{\color{red}4}) & ({\color{blue}3},{\color{red}5})\\
%\hline
%{\color{blue}Bob} go to No.1 & ({\color{blue}0},{\color{red}10}) & ({\color{blue}7},{\color{red}0})\\
%\hline
%\hline
%\end{tabular}
%\begin{itemize}
%\item When the two persons are stuck in this dilemma, a third person comes out and say, {\Large "Why don't you just choose the restaurant by probability?}"
%\item That's it!
%\end{itemize}
%\end{frame}
%
%\begin{frame}
%\frametitle{What should they do?}
%\begin{tabular}{|c|c|c|}
%\hline
%\hline
%    & {\color{red}Alice} go to No.3 & {\color{red}Alice} go to No.1\\
%\hline
%{\color{blue}Bob} go to No.3 & ({\color{blue}10},{\color{red}4}) & ({\color{blue}3},{\color{red}5})\\
%\hline
%{\color{blue}Bob} go to No.1 & ({\color{blue}0},{\color{red}10}) & ({\color{blue}7},{\color{red}0})\\
%\hline
%\hline
%\end{tabular}
%\begin{itemize}
%\item Supposed that Alice will choose No.3 by probability $a_1$ and choose No.1 by $a_2$.
%\end{itemize}
%\end{frame}
%
%\begin{frame}
%\frametitle{What should they do?}
%\begin{tabular}{|c|c|c|}
%\hline
%\hline
%    & {\color{red}Alice} go to No.3 & {\color{red}Alice} go to No.1\\
%\hline
%{\color{blue}Bob} go to No.3 & ({\color{blue}10},{\color{red}4}) & ({\color{blue}3},{\color{red}5})\\
%\hline
%{\color{blue}Bob} go to No.1 & ({\color{blue}0},{\color{red}10}) & ({\color{blue}7},{\color{red}0})\\
%\hline
%\hline
%\end{tabular}
%\begin{itemize}
%\item Supposed that Alice will choose No.3 by probability $a_1$ and choose No.1 by $a_2$.
%\item Then if Bob go to No.3, his pay-off will be $10a_1+3a_2$. If he go to No.1, then pay-off will be $7a_2$.
%\end{itemize}
%\end{frame}
%
%\begin{frame}
%\frametitle{What should they do?}
%\begin{tabular}{|c|c|c|}
%\hline
%\hline
%    & {\color{red}Alice} go to No.3 & {\color{red}Alice} go to No.1\\
%\hline
%{\color{blue}Bob} go to No.3 & ({\color{blue}10},{\color{red}4}) & ({\color{blue}3},{\color{red}5})\\
%\hline
%{\color{blue}Bob} go to No.1 & ({\color{blue}0},{\color{red}10}) & ({\color{blue}7},{\color{red}0})\\
%\hline
%\hline
%\end{tabular}
%\begin{itemize}
%\item Supposed that Alice will choose No.3 by probability $a_1$ and choose No.1 by $a_2$.
%\item Then if Bob go to No.3, his pay-off will be $10a_1+3a_2$. If he go to No.1, then pay-off will be $7a_2$.
%\item $10a_1+3a_2$ must be equal to $7a_2$, otherwise Bob can decide indeed which restaurant to go. And it will be circulation again.\\
%\end{itemize}
%\end{frame}
%
%\begin{frame}
%\frametitle{What should they do?}
%        \qquad $10a_1+3a_2$ = $7a_2$ \\
%        \qquad $a_1 + a_2 = 1$ \\
%        \qquad So $a_1 = \frac{2}{7}, a_2 = \frac{5}{7}$
%\end{frame}
%
%
%\begin{frame}
%\frametitle{What should they do?}
%\begin{tabular}{|c|c|c|}
%\hline
%\hline
%    & {\color{red}Alice} go to No.3 & {\color{red}Alice} go to No.1\\
%\hline
%{\color{blue}Bob} go to No.3 & ({\color{blue}10},{\color{red}4}) & ({\color{blue}3},{\color{red}5})\\
%\hline
%{\color{blue}Bob} go to No.1 & ({\color{blue}0},{\color{red}10}) & ({\color{blue}7},{\color{red}0})\\
%\hline
%\hline
%\end{tabular}
%\begin{itemize}
%\item Supposed that Bob will choose No.3 by probability $b_1$ and choose No.1 by $b_2$.
%\item Based on the same method, we can get that $b_1 = \frac{10}{11},b_2 = \frac{1}{11}$
%\item If both of them choose based on the probability, then it is a equilibrium.
%\end{itemize}
%\end{frame}
%
%\begin{frame}
%\frametitle{More Complicate situation.}
%{\color{red}Alice} can choose from $(x_1,x_2,...,x_i)$, and the probability vector will be $\overrightarrow{x}$.\\
%{\color{blue}Bob} can choose from $(y_1,y_2,...,y_j)$, and the probability vector will be $\overrightarrow{y}$.\\
%{\color{red}A}($\overrightarrow{x}$,$\overrightarrow{y}$) means {\color{red}Alice}'s paid-off.\\
%{\color{blue}B}($\overrightarrow{x}$,$\overrightarrow{y}$) means {\color{blue}Bob}'s paid-off.\\
%\end{frame}
%
%\begin{frame}
%\frametitle{More Complicate situation.}
%Suppose {\color{red}Alice} plays $\overrightarrow{x}$.\\
%Can {\color{blue}Bob} do better than {\color{blue}B}($\overrightarrow{x}$,$\overrightarrow{y}$)?\\
%That is:\\
%        \qquad $\exists \overrightarrow{v} ~~s.t. ~~{\color{blue}B}(\overrightarrow{x},\overrightarrow{v}) >{\color{blue}B}(\overrightarrow{x},\overrightarrow{y})$?
%\end{frame}
%
%\begin{frame}
%\frametitle{Nash Equilibrium}
%DEF:\\
%\qquad $\overrightarrow{x}$,$\overrightarrow{y}$ is a Nash equilibrium if\\
%\qquad $\forall \overrightarrow{u} ~~{\color{red}A}(\overrightarrow{x},\overrightarrow{y}) \geq{\color{red}A}(\overrightarrow{u},\overrightarrow{y})$\\
%\qquad $\forall \overrightarrow{v} ~~{\color{blue}B}(\overrightarrow{x},\overrightarrow{y}) \geq{\color{blue}B}(\overrightarrow{x},\overrightarrow{v})$
%\end{frame}
%
%
%\section{Nash's Theorem}
%\begin{frame}{Nash's Theorem}
%	\begin{itemize}[<+->]
%		\item \textbf{\large Nash's Theorem}:
%		
%		\qquad Every game with a finite number of players and a finite number of actions
%		available to each player has a Nash equilibrium.
%		\item As for Bob and Alice, there must be a point that they won't change their strategies.
%	\end{itemize}
%\end{frame}
%
%\begin{frame}[fragile]{How to prove it?}
%\begin{itemize}[<+->]
%	\item Nash's original proof of it used \textbf{Kakutani's fixed point theorem}.
%	\item But a year later Nash simplified his proof to only use \textbf{Brouwer's fixed point theorem}.
%\end{itemize}
%\end{frame}
%
%\begin{frame}[fragile]{How to prove it?}
%	\begin{itemize}
%		\item Nash's original proof of it used \textbf{Kakutani's fixed point theorem}.
%		\item But a year later Nash simplified his proof to only use \textbf{\color{red}\large Brouwer's fixed point theorem}.
%	\end{itemize}
%\end{frame}
%
%\begin{frame}[fragile]{Brouwer's fixed point theorem}
%	\begin{itemize}
%		\item \textbf{\large Brouwer's fixed point theorem}:
%		
%		\qquad Let $D$ be a convex, compact subset of the Euclidean space. If $f : D\  \overrightarrow{}\ D$ is
%		continuous, then there exists $x \in D$ such that $f (x) = x$.
%	\end{itemize}
%\end{frame}
%
%\begin{frame}[fragile]{Brouwer's fixed point theorem}
%	\begin{itemize}
%		\item \textbf{Examples}:
%		\begin{itemize}
%			\item Take an ordinary map of a country, and suppose that that map is laid out on a table inside that country. There will always be a "You are Here" point on the map which represents that same point in the country.
%		\end{itemize}
%	\end{itemize}
%\end{frame}
%
%\begin{frame}[fragile]{Gain function}
%	\begin{itemize}[<+->]
%		\item Now we introduce the idea of \textbf{Gain function}:\\
%		\begin{equation}
%		{\color{blue}Gain_{Bob}} ({\color{blue}\overrightarrow{x}}, {\color{red}\overrightarrow{y}}, {\color{blue}i}) = max\{{\color{blue}B}({\color{blue}\overrightarrow{e_i}}, {\color{red}\overrightarrow{y}})-{\color{blue}B}({\color{blue}\overrightarrow{x}}, {\color{red}\overrightarrow{y}}), 0\} \ \  \ \ \ \ \ \nonumber
%		\end{equation}
%		\begin{equation}
%		{\color{red}Gain_{Alice}} ({\color{blue}\overrightarrow{x}}, {\color{red}\overrightarrow{y}}, {\color{red}j}) = max\{{\color{red}A}({\color{blue}\overrightarrow{x}}, {\color{red}\overrightarrow{e_j}})-{\color{red}A}({\color{blue}\overrightarrow{x}}, {\color{red}\overrightarrow{y}}), 0\} \ \  \ \ \ \ \ \nonumber
%		\end{equation}
%		\item In other words, the $Gain$ is equal to the increase in payoff for a player if he were to switch to another strategy.
%		\item Obviously, the $Gain$ for all players is 0 in \textbf{Nash Equilibrium}.
%	\end{itemize}
%\end{frame}
%
%\begin{frame}[fragile]{Proof of Nash' Theorem}
%	\begin{itemize}[<+->]
%		\item Now we define a function as follows:
%		 \begin{equation}
%		 {\color{blue}f}({\color{blue}\overrightarrow{x}}, {\color{red}\overrightarrow{y}}, {\color{blue}i}) = \frac{{\color{blue}x_i} + {\color{blue}Gain_{Bob}} ({\color{blue}\overrightarrow{x}}, {\color{red}\overrightarrow{y}}, {\color{blue}i})}{1 + \sum_i{\color{blue}Gain_{Bob}} ({\color{blue}\overrightarrow{x}}, {\color{red}\overrightarrow{y}}, {\color{blue}i})}\ \  \ \ \ \ \ \nonumber
%		 \end{equation}
%		 \begin{equation}
%		 {\color{red}g}({\color{blue}\overrightarrow{x}}, {\color{red}\overrightarrow{y}}, {\color{red}j}) = \frac{{\color{red}y_j} + {\color{red}Gain_{Alice}} ({\color{blue}\overrightarrow{x}}, {\color{red}\overrightarrow{y}}, {\color{red}j})}{1 + \sum_j{\color{red}Gain_{Alice}} ({\color{blue}\overrightarrow{x}}, {\color{red}\overrightarrow{y}}, {\color{red}j})}\ \  \ \ \ \ \ \nonumber
%		 \end{equation}
%		\item In other words, function $f$ and $g$ tries to boost the probability mass that player places on various pure strategies
%		depending on the each one's gains in payoff the player would get by switching to these strategies.
%	\end{itemize}
%\end{frame}
%
%\begin{frame}[fragile]{Proof of Nash' Theorem}
%	\begin{itemize}[<+->]
%		\item These function is a map from a 2-dimension space to a 2-dimension space.
%		\begin{equation}
%		(\overrightarrow{x}, \overrightarrow{y}) \overrightarrow{}(\overrightarrow{x'}, \overrightarrow{y'})\ \  \ \ \ \ \  \nonumber
%		\end{equation}
%	\end{itemize}
%\end{frame}
%
%\begin{frame}[fragile]{Proof of Nash' Theorem}
%	\begin{itemize}[<+->]
%		\item It is easy to see that this function is continuous. So we can use \textbf{Brouwer's fixed point theorem}, there is at least one fixed point of the function.
%		\item For any fixed point
%		\begin{equation}
%		{\color{blue}Gain_{Bob}} ({\color{blue}\overrightarrow{x}}, {\color{red}\overrightarrow{y}}, {\color{blue}i}) = 0, \ \ \  \forall i\in [n] \nonumber
%		\end{equation}
%		\begin{equation}
%		{\color{red}Gain_{Alice}} ({\color{blue}\overrightarrow{x}}, {\color{red}\overrightarrow{y}}, {\color{red}j}) = 0, \ \ \  \forall j\in [n]\nonumber
%		\end{equation}
%		It can be proved by by contradiction.
%		\item Then we claim that any fixed point of this function is a \textbf{Nash equilibrium}.
%	\end{itemize}
%\end{frame}
%
%\begin{frame}[fragile]{The smile of John Nash}
%\end{frame}


\begin{frame}
\Huge{\centerline{The End}}
\end{frame}
\end{document}
